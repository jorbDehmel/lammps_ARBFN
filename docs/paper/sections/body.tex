
\subsection{\texttt{fix arbfn}}

\end{multicols}
\begin{figure}[H]
    \centering
    \includegraphics[width=0.95\textwidth]{arbfn_protocol}
    \caption{\texttt{fix arbfn} protocol.}
    \label{fig:arbfn_protocol}
\end{figure}
\begin{multicols}{2}

\subsection{\texttt{fix arbfn/ffield}}

\end{multicols}
\begin{figure}[H]
    \centering
    \includegraphics[width=0.95\textwidth]{ffield_protocol}
    \caption{\texttt{fix arbfn/ffield} protocol.}
    \label{fig:ffield_protocol}
\end{figure}
\begin{multicols}{2}

Although the difference between \ref{fig:arbfn_protocol} and
\ref{fig:ffield_protocol} may seem trivial, the omission of the
controller from the simulation loop allows
\ref{fig:ffield_protocol} to run orders of magnitude faster.

\subsection{Running Simulations}

In order to keep ARBFN MPI communication from interfering with
internal LAMMPS communication, we must run LAMMPS with the
\texttt{-mpicolor ...} command-line argument. The number
following this must be anything \textbf{except $56789$}. On
Linux systems, this takes the form that follows.

\begin{code}
mpirun -n 1 ./controller : \
    -n ${num_lmp_threads} \
    lmp -mpicolor 123 \
    -in input_script.lmp
\end{code}
