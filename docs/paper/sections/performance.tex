
% Setup
Our performance experiments were carried out with small
simulations. We kept a constant particle density of 2 atoms per
unit area in a 2D system (where the square box's edge length was
varied) to avoid minimization problems arising from varying
particle directly. These experiments are intended to demonstrate
the trend of the running times in small simulations, as well as
to corroborate that our fixes scale proportionally to a no-fix
system. A \texttt{Python} script was used to automate the
running of scripts.

\begin{figure}
    \centering
    \begin{subfigure}{.25\textwidth}
        \centering
        \includegraphics[width=\textwidth]{arbfn_comparison}
    \end{subfigure}%
    \begin{subfigure}{.25\textwidth}
        \centering
        \includegraphics[width=\textwidth]{
            arbfn_scaled_comparison}
    \end{subfigure}

    \caption{
        Comparing \texttt{fix arbfn} to control and
        \texttt{fix arbfn/ffield}. While still approximately
        linear with respect to the number of atoms, it runs
        \emph{much} slower than the latter two (which are about
        the same speed).
    }
    \label{fig:arbfn_comparison}
\end{figure}

% Results
Figure \ref{fig:arbfn_comparison} demonstrates the sharp slope
of \texttt{fix arbfn}: IPC is extremely costly, and
frame-by-frame updates should be avoided whenever possible. That
being said, the time usage appears to scale proportionally to
control as expected. The \texttt{fix arbfn/ffield} results
appear to scale with a much smaller coefficient, as expected:
This is a much more usable fix, although its usecase is more
narrow.
